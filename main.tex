\documentclass[12pt, a4paper]{article}

% --- Paquetes Esenciales ---
\usepackage[utf8]{inputenc}
\usepackage[spanish]{babel}
\usepackage{amsmath}
\usepackage{graphicx}
\usepackage{geometry}
\usepackage{hyperref}

% --- Configuración de la Bibliografía ---
\usepackage[backend=biber, style=apa]{biblatex}
\addbibresource{referencias.bib}

% --- Configuración de los Márgenes ---
\geometry{a4paper, margin=2.5cm}

% --- Metadatos del Artículo ---
\title{Título de Nuestra Investigación}
\author{
  Primer Autor\\
  \small Universidad Nacional del Altiplano, Puno\\
  \small \texttt{correo1@ejemplo.com}
  \and
  Segundo Autor\\
  \small Afiliación\\
  \small \texttt{correo2@ejemplo.com}
}
\date{6 de agosto de 2025}

% ===================================================================
% DOCUMENTO
% ===================================================================
\begin{document}

\maketitle

\begin{abstract}
\noindent
Aquí va el resumen del artículo (abstract). Debe ser un párrafo conciso que describa el objetivo, la metodología, los resultados clave y la conclusión principal de su investigación.
\end{abstract}

\tableofcontents
\newpage

\section{Introducción}
Aquí se presenta el contexto del problema y se citan trabajos previos, como un artículo \cite{autor2025articulo} o un libro \cite{autor2024libro}. El objetivo del estudio se declara al final de esta sección.

\section{Metodología}
Se describe en detalle cómo se realizó la investigación. Es común usar subsecciones para organizar la información sobre la recolección de datos y el análisis.

\section{Resultados}
Se presentan los hallazgos de manera objetiva, sin interpretación. Es ideal usar tablas (Cuadro \ref{tab:ejemplo}) y figuras (Figura \ref{fig:ejemplo}) para visualizar los datos.

% --- Ejemplo de Figura ---
\begin{figure}[h!]
  \centering
  % Para usar esto, pon una imagen en la carpeta "imagenes" y cambia el nombre del archivo
  % \includegraphics[width=0.7\textwidth]{imagenes/tu_grafico.png}
  \caption{Título descriptivo de la figura.}
  \label{fig:ejemplo}
\end{figure}

% --- Ejemplo de Tabla ---
\begin{table}[h!]
  \centering
  \begin{tabular}{|l|c|r|}
    \hline
    Variable de Estudio & Medida A & Medida B \\
    \hline
    Grupo 1 & 123 & 45.6 \\
    Grupo 2 & 456 & 78.9 \\
    \hline
  \end{tabular}
  \caption{Título descriptivo de la tabla.}
  \label{tab:ejemplo}
\end{table}

\section{Discusión}
Aquí se interpretan los resultados. ¿Qué significan? ¿Cómo se comparan con los de otros estudios? También se mencionan las limitaciones del trabajo.

\section{Conclusiones}
Se